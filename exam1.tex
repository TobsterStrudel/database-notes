\documentclass[12pt]{report}
\title{\textbf{COP3703 Intro to Databases\\Exam 1 Notes}}
\author{Tobias Dault}
\date{\today}

\addtolength{\textheight}{3cm}
\usepackage{enumitem}
\usepackage{hyperref}
\usepackage{float}
\usepackage{graphicx}
\usepackage{multicol}
\usepackage{subcaption}
\usepackage[tmargin=1in,lmargin=1in,rmargin=1in]{geometry}
\usepackage{titlesec}

\DeclareCaptionFormat{custom}
{%
	\textbf{\footnotesize #1#2}\textit{\footnotesize #3}
}
\captionsetup{format=custom}

\graphicspath{ {./assets/} }

\hypersetup{ colorlinks=true, linkcolor=blue, filecolor=purple, urlcolor=cyan }

\newlength\tindent
\setlength{\tindent}{\parindent}
\setlength{\parindent}{0pt}
\renewcommand{\indent}{\hspace*{\tindent}}

\setlist[description]{noitemsep, topsep=0pt, itemsep=.5em}
\setlist[enumerate]{noitemsep, topsep=0pt, itemsep=.5em}
\setlist[itemize]{noitemsep, topsep=0pt, itemsep=.5em}

\titleformat{\chapter}
{\Large\bfseries}
{\thechapter.}{0.5em}{}

\titleformat{\section}
{\large\bfseries}
{\thesection.}{0.5em}{}

\titleformat{\subsection}
{\normalsize\bfseries}
{\thesubsection.}{0.5em}{}

\titlespacing\chapter{0pt}{12pt plus 0pt minus 4pt}{0pt plus 0pt minus 4pt}
\titlespacing\section{0pt}{12pt plus 4pt minus 8pt}{0pt plus 2pt minus 8pt}
\titlespacing\subsection{0pt}{12pt plus 4pt minus 2pt}{0pt plus 2pt minus 2pt}

\begin{document}
	
	\maketitle
	\tableofcontents
	\thispagestyle{empty}
	
	\chapter{Week 1}
	\section{Definitions}
	
	\begin{description}[style=multiline,leftmargin=12em]
		\item [Database] a collection of related data with implicit meaning.
		\item [DBMS] Database management system. A software package/system to facilitate 
		the creation and maintenance of a computerized database that takes care of storage, query execution, and database manipulation.
		\item [DBA] Database Administrator
		\item [Database System] The DBMS software together with the data itself. Sometimes, the applications are also included.
		\item [Data Records] Rows or columns of data with correlated data.
		\item [Data Elements] Elements within a database, like cells in Excel.
		\item [Data Types] Integer, Varchar(255), etc.
		\item [Program-data Independence] Insulation between programs and data. Allows changing data structures and storage organization without having to change the DBMS access programs.
		\item[Data Abstraction] Conceptual data representation different from data storage. Words and recognizable symbols rather than the raw ones and zeros of low level computation.
		\item[Actors] Everything interacting with the database system: DBMS designers, system admins, database admins, users, apps, data source, etc.
	\end{description}
	
	\section{Database Attributes}
	\begin{description}
		\item [Self-describing]
		\item [Insulation]
		\item [Support of multiple user views]
	\end{description}
	
	\chapter{Week 2}
	\section{Definitions}
	\begin{description}[style=multiline,leftmargin=12em]
		\item [Schemas/Schemata] Data (record) structure, untyped, no constraints or relationships, and doesn't change frequently. \textit{Description of data}.
		\item [Schema Evolution] the change of a schema (avoid it). Could indicate poor design if required.
		\item [Schema Construct] Specific names of schemata (table headers/titles).
		\item [Database Snapshot] Data of databases at a specific time.
		\item [Empty State] Only the schema, \textit{no population}.
		\item [Initial State] When the database is first populated
		\item [3-layer Architecture] External views, conceptual schema, and internal schema. Recommended format. Also called 3-schema/ANSI architecture.
		\item [External View] External/Conceptual mapping. End users. Provides separate views to different users, Permits only certain operations: \textbf{submit form data}, \textbf{execute stored procedures}.
		\item [Conceptual Schema] Conceptual/Internal mapping. Hides the details of the physical storage structure. \textbf{Entities}, \textbf{Attributes}, and \textbf{Relationships}.
		\item [Internal Schema] Contains actual data, stored database. Describes the physical storage structure. Contains \textbf{access path} information. 
		\item [Logical data independece] change the conceptual schema without having to change external schemata or application programs.
		\item [Physical data independece]change the internal schema without having to change the conceptual schema.
		
		\item [Submit form data]
		\item [Stored Procedures]
		\item [Entities] represent real world objects. Ex. Students, Courses
		\item [Attributes] Represent entity properties. Ex. Name, ID
		\item [Relationships] Represent entity associations. Ex. Is enrolled in
		\item [Access paths] Search structure that allows finding of records by attributes, uses indexing or hashing, may be optimized for different common-case queries.
		
		\item [DBMS Languages] \textbf{DDL}, \textbf{SDL}, \textbf{DML} all used in SQL. These languages are usually not considered distinct languages.
		\item [DDL] Data definition language. Conceptual and internal schemata.
		\item [SDL] Storage definition language. Internal schema
		\item [DML] Data manipulation language. A means for users to manipulate the database. Retrieval, insertion, deletion, and modification of the data.
		
		
		\item [Concurency Control]
	\end{description}
	\section{Database Creation}
	Every time the database is updated, we get another database state i.e. the current state (current snapshot)
	
	
\end{document}