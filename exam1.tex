\documentclass[12pt]{report}
\title{\textbf{COP3703 Intro to Databases\\Exam 1 Notes}}
\author{Furio. Digitized by Tobias Dault}
\date{\today}

\addtolength{\textheight}{3cm}
\usepackage{enumitem}
\usepackage{hyperref}
\usepackage{float}
\usepackage{graphicx}
\graphicspath{ {./images/} }
\usepackage{multicol}
\usepackage{subcaption}
\usepackage{tikz}
\usepackage[tmargin=1in,lmargin=1in,rmargin=1in]{geometry}
\usepackage{titlesec}

\DeclareCaptionFormat{custom}
{%
	\textbf{\footnotesize #1#2}\textit{\footnotesize #3}
}
\captionsetup{format=custom}
\hypersetup{ colorlinks=true, linkcolor=blue, filecolor=purple, urlcolor=cyan }

\newlength\tindent
\setlength{\tindent}{\parindent}
\setlength{\parindent}{0pt}
\renewcommand{\indent}{\hspace*{\tindent}}
\newcommand*\mycirc[1]{%
	\begin{tikzpicture}
		\node[draw,circle,inner sep=1pt] {#1};
\end{tikzpicture}}
\setlist[description]{noitemsep, topsep=0pt, itemsep=.5em}
\setlist[enumerate]{noitemsep, topsep=0pt, itemsep=.5em}
\setlist[itemize]{noitemsep, topsep=0pt, itemsep=.5em}

\titleformat{\chapter}
{\Large\bfseries}
{\thechapter.}{0.5em}{}

\titleformat{\section}
{\large\bfseries}
{\thesection.}{0.5em}{}

\titleformat{\subsection}
{\normalsize\bfseries}
{\thesubsection.}{0.5em}{}



\begin{document}
	
	\maketitle
	\tableofcontents
	\thispagestyle{empty}
	
	\chapter{Chapter 1}
	\section{Why Databases?}
	\begin{itemize}
		\item Goal of reliable info, storage, retrieval, and assurance at scale.
		\item Definition: A collection of data that is related and has an implicit meaning.
		\item \textbf{Implicit Properties}
		\begin{itemize}
			\item represents "real-world" aspect (entities), where changes to this subset of the real world (UoD) show on the database.
			\item data is logically coherent with purpose for users or programs.
		\end{itemize}
		\item \textbf{Why databases $>$ Files (in some cases)}
		\begin{itemize}
			\item consider two offices that need similar data
			\item we should give them a way to simultaneously access and update the same data, to avoid redundancies and wasted space.
		\end{itemize}
	\end{itemize}
	\section{Database Management System (DBMS)}
	\begin{itemize}
		\item software for creation and maintenance of computer database.
		\item manages storage, queries, and data manipulation.
		\item this class uses Oracle DBMS
	\end{itemize}
	\section{Database System}
	\begin{itemize}
		\item The DBMS software as well as the actual data, and potentially other applications.
		\item Where you actually query, as well as what defines the "database definition" (metadata).
	\end{itemize}
	
	\section{Terminology/Example Database}
	\begin{center}
		\begin{tabular}{|c|c|c|c|}
			\hline
			Name & Number & Class & Major\\
			\hline
			Tony & 123 & IT30 & IT\\
			Robert & 124 & CS20 & CS\\
			\hline
		\end{tabular}
		\\ $\ast$ one table in the database, also includes course, grades, pre-req tables, etc.$\ast$
	\end{center}
	\begin{itemize}
		\item any row is a \underline{record}
		\item any single item is an \underline{element}
		\item all elements need a particular \underline{type}, i.e. int or char.
		\item may need to combine info from multiple tables to satisfy a \underline{query}
		\item queries provide \underline{information} to the user, ideally useful.
	\end{itemize}
	\section{Self-Describing Data Structure}
	\begin{itemize}
		\item relations (the aforementioned tables) need to have their number of columns identified, and columns must have their type and associated relation specified.
	\end{itemize}
	\section{Program-data Independence}
	\begin{itemize}
		\item Allows changes to data structures and storage without changing DBMS access programs.
		\item Basically, API function access over direct access.
	\end{itemize}
	\section{Data Abstraction}
	\begin{itemize}
		\item conceptual data representation abstract details of data storage.
	\end{itemize}
	\section{Support for Multiple Views}
	\begin{itemize}
		\item Relations may be combined and presented in a relevant and useful way to users (i.e., GUIs for general works, hiding SSNs from non-HR).
	\end{itemize}
	\section{Multi-user Transactions}
	\begin{itemize}
		\item Transactions may be completed reliably, concurrency is accounted for (i.e. no double-booking).
		\item For any transaction in a complex system, there are likely to be many complex steps and queries.
	\end{itemize}
	\section{Bad use Case}
	\begin{itemize}
		\item Databases require extra hardware software, training, and implement costly features to help concurrency, ease of access.
	\end{itemize}
	\chapter{Chapter 2}
	\section{Database Schema/Schemata}
	\begin{itemize}
		\item data/record structure
		\item NO TYPE INFO!
		\item no constraints/relations
		\item does not changes frequently (evolution)
	\end{itemize}
	\section{Database State}
	\begin{itemize}
		\item data of database at any given time is called its \underline{state}, or a \underline{snapshot}
		\item \textbf{Empty} - just the schema
		\item \textbf{Initial State} after first populated data
		\item state changes after each update.
		\item bad for static data, low-resource/embedded systems, real-time requirements, or single-user.
	\end{itemize}
	$\ast$INSERT EXAMPLE HERE$\ast$
	
	\section{Design Goals}
	\begin{itemize}
		\item self describing
		\item insulation of programs and data
		\item multi-user access/view
	\end{itemize}
	\section{Brainstorming-Building a Databases}
	\begin{itemize}
		\item data is well labeled and related
		\item mechanism to prevent simultaneous modification
		\item buffer breaking connection ????? outside access and stored data
	\end{itemize}
	\section{Three Schema Architecture (ANSI Architecture)}
	$\ast$INSERT DIAGRAM$\ast$
	\section{Internal Schema}
	\begin{itemize}
		\item describes physical storage
		\item access paths
		\begin{itemize}
			\item search structure using indices, trees, hashing, etc.
			\item ideally optimized for common case.
		\end{itemize}
	\end{itemize}
	\section{Conceptual Schema}
	\begin{itemize}
		\item hides details of physical storage
		\item introduces entities, attributes, relationships, to describe and model data more naturally.
	\end{itemize}
	\section{External Schema}
	\begin{itemize}
		\item provides different views as appropriate to various user type (access control)
		\item what process type of users are permitted to execute
		\item get form data
	\end{itemize}
	\section{Data Independence}
	\begin{itemize}
		\item \textbf{Logical data independence}: ability to change conceptual schema without change to external schema
		\item \textbf{Physics data independence}: change internal without change to conceptual schema
	\end{itemize}
	\section{DBMS Languages}
	\begin{itemize}
		\item Data Definition Language (DDL)
		\item Storage Definition Language (SDL)
		\item Data Manipulation Language (DML)
		\item in modern systems, all in one language but separately compiled.
	\end{itemize}
	\section{DBMS}
	\begin{itemize}
		\item common (keyboard, forms, speech to text) to less considered (embedded system, cameras)
	\end{itemize}
	\section{Evolution of Databases/DBMSs}
	\begin{itemize}
		\item intrinsically linked to growth of computers
		\item \textbf{Centralized}: early on, runs as a normal program on non-networked computer $\rightarrow$ singlepoint failure.
		\item \textbf{Two-tier}: client linked to server via network, but no buffer $\rightarrow$ no program/data insulation
		\item \textbf{Three-tier}: modern external/conceptual/internal design
	\end{itemize}
	\chapter{Chapter 3}
	\section{Basics of ER Diagrams}
	\begin{itemize}
		\item \textbf{Entity}: something in the miniworld with an independent existence.
		\item \textbf{Attribute}: properties that describe the entity
		\begin{itemize}
			\item \underline{Atomic}: an attribute which cannot be further divided
			\item \underline{Compound}: an attribute consisting of other compound/atomic attributes (design choice)
			\item \underline{single} vs \underline{multivalued} possible (i.e. dropdown vs checkboxes, also a design decision)
			\item \underline{Stored} (i.e. directly storing age) \underline{derived} (i.e. getting age from birthday)
			\item \underline{NULL} values (interpret as N/A, missing, etc)
			\item \underline{Complex} attributes, such as nested, composite, or multivalued attributes.
			\item \underline{Key Attribute} Can ID each entity uniquely as efficiently as possible due to two properties.
			\begin{itemize}
				\item Uniqueness (within entity type)
				\item Minimality (i.e. why store name and n-number when n-number is already unique)
			\end{itemize}
		\end{itemize}
		
	\end{itemize}
	\includegraphics[scale=0.5]{image1.png}
	\section{Entity Type}
	\begin{itemize}
		\item a name representing the set of all entities of the same type (always singular, all caps)
		\item \textbf{Intention/schema}: attribute types associated with entity
		\item \textbf{Extension/entity set}: all entities in the DB of this particular type.
		\item \textbf{n-tier architecture}: scale increases, complex platform support, difficult for one layer to handle all abstraction, provide useful APIs for each layer
	\end{itemize}
	\section{Exercise}: Web system for airline tickets/reservation\\
	$\ast$INSERT VISUAL$\ast$
	\section{Weak Entity Identification}
	\begin{itemize}
		\item double line
		\item a weak entity can be uniquely identified by combining its partial key with the key of the entity on the other side of the double-lined relationship this weak entity is connected to, by double lines.
	\end{itemize}
	\chapter{Chapter 4}
	\section{Extended Entity Relationship (EER) Diagrams}
	\begin{itemize}
		\item Superset of ER diagrams, primarily adding subclasses and superclasses, plus union and category types with inheritance.
	\end{itemize}
	\subsection{Super class $-->--$ subclass}
	\begin{itemize}
		\item subclass inherits all attributes of parent, and may define additional ones
		\item relationships may be defined on subclasses and not superclasses as well
		\begin{itemize}
			\item but, a subclass inherits relationships from the parent
		\end{itemize}
	\end{itemize}
	\section{Specialization}
	\begin{itemize}
		\item \textbf{specialization} is the process of defining a set of subclasses of an entity type
		\item defined on basis of distinguishing characteristic in entity type.
	\end{itemize}
	\section{Predicate vs. User Defined Subclasses}
	\begin{itemize}
		\item \textbf{Predicate Defined} if membership is determined by placing condition on value of attribute of superclass
		\item \textbf{User Defined} if membership is specified for all entities by the user.
	\end{itemize}
	\section{Disjoint vs. Overlapping}
	\begin{itemize}
		\item \textbf{Disjoint} if entity may not belong to $>1$ subclass
		\begin{itemize}
			\item indicated by \mycirc{d} branch off superclass
		\end{itemize}
		\item \textbf{Overlapping} if there is no disjointness constraint
		\begin{itemize}
			\item indicated by \mycirc{o}
		\end{itemize}
	\end{itemize}
	\section{Complete (total) vs Partial}
	\begin{itemize}
		\item \textbf{complete}  if entity must belong to a subclass. The union of all entities in its superclasses.
		\begin{itemize}
			indicated by double line
		\end{itemize}
		\item \textbf{partial} if there is not a completeness requirement. Subset of the union
	\end{itemize}
	\section{Specialization Hierarchy vs Lattice}
	\begin{itemize}
		\item \textbf{hierarchy} if every subclass is allowed to be a subclass of only one parent class $\rightarrow$ tree/strict hierarchy
		\item \textbf{lattice} if each subclass is permitted to be a subclass of $>1$ parent
		\begin{itemize}
			\item same idea as multiple inheritance, and subclass must have parent from all classes it inherits
		\end{itemize}
	\end{itemize}
	\section{UNION type}
	\begin{itemize}
		\item represents collection of entities from different types
		\item a union of entities is called a \textbf{uniontype} or \textbf{category}
		\item ex. bank$\cup$person$\cup$company$\rightarrow$owner
		\item category inherits attributes only of the subclass it belongs to
		\item needed for situations where "is all" relationship does not exist
	\end{itemize}
	\section{ERR Design Guidelines}
	\begin{itemize}
		\item only use subclasses when deemed necessary
		\item union/categories should generally be avoided
		\item choice of disjoint/overlapping and user/predicate defined should be based on the miniworld requirements.
		\begin{itemize}
			\item default to overlapping/partial
		\end{itemize}
	\end{itemize}
	
	\chapter{Chapter 9}
	\section{ER-to-Relation}
	\begin{itemize}
		\item \textbf{Step 1}: Map Regular Entity Type
		\begin{itemize}
			\item for each regular (strong) entity type $E$, create relation $R$ with all of $E$'s sample attributes (flatten composites)
			\item called "entity relations", and each tuple is "entity instance"
			\item choose ONE key in $E$ to be the Private Key in $R$.
		\end{itemize}
		\item \textbf{Step 2: Map Weak Entity Types}
		
	\end{itemize}
	
	
	
	
\end{document}